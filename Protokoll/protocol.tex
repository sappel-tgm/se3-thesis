\documentclass[]{protocol}
% Required
\title{Funktionieren Smart Contracts als vertragliche Vereinbarungen?}
\author{Simon Appel}
% Optional
\mysubtitle{Laborprotokoll}
\mysubject{Systemtechnik Labor}
\mycourse{5xHIT 2018/19, Gruppe A}
% Version
\myteacher{PAWD,BORM}
\myversion{0.1}
\mybegin{05.10.18}
\myfinish{}
% \setcode{frame=single} 			% Add a frame to codes
% \setcode{bgcolor=AlmostWhite}		% Add a background to codes (minted only)
% \usemintedstyle{trac} 	
		% autumn, rainbow_dash, tango (default), trac
\begin{document}
% \thispagestyle{fancy}				% Makes the first page fancy too
% \begin{abstract}\end{abstract} 	% Add a short overview

\section{Grobgliederung}
\subsection{Einleitung}
Einleitung zu dem Thema
\subsubsection{Was sind Smart Contracts}
Hier werden Smart Contracts erklärt bzw. verlinkt. 
\subsection{Anwendungsbereiche}
Einsatzgebiete die es schon gibt
\subsubsection{Juristischer Einsatz}
\subsubsection{Business Einsatz}
\subsection{Vergleich}
Vorteile/Nachteile vergleichen. (Bessere Überschrift)
\subsubsection{Vorteile}
\subsubsection{Nachteile}
\subsection{Ergebnis}
Antwort auf die Fragestellung

\section{Quellen}
\begin{itemize}
	\item https://link.springer.com/article/10.1007/s00287-017-1045-2
	\item https://link.springer.com/article/10.1365/s40702-018-00445-x
	\item https://link.springer.com/article/10.1007/s00502-016-0431-9
	\item https://link.springer.com/chapter/10.1007/978-3-658-20265-1-9
	\item https://link.springer.com/chapter/10.1007/978-3-662-55890-4-18
	\item https://link.springer.com/chapter/10.1007/978-3-658-02844-2-15
	\item https://link.springer.com/chapter/10.1007/978-3-658-14106-6-2
	\item https://link.springer.com/chapter/10.1007/978-3-658-20199-9-5
	\item http://katalog.ub.tuwien.ac.at/AC15022342
\end{itemize}
Einige Quellen werden trotz Verbindung nicht kostenlos angeboten.
Bei den letzten 5 Quellen, den letzten Bindestrich mit Unterstrich ersetzen.

% \glsaddall 		% Add all glossary entries to printglossaries
\end{document}
